\documentclass[modern]{aastex61}
\usepackage{graphicx}
\usepackage{xcolor}
\usepackage[sort&compress]{natbib}
\usepackage[hang,flushmargin]{footmisc}

% units macros
\newcommand{\unit}[1]{\mathrm{#1}}
\newcommand{\km}{\unit{km}}
\newcommand{\m}{\unit{m}}
\newcommand{\s}{\unit{s}}
\newcommand{\kms}{\km\,\s^{-1}}
\newcommand{\ms}{\m\,\s^{-1}}
\newcommand{\ang}{\text{\normalfont\AA}}

% math macros
\newcommand{\dd}{\mathrm{d}}
\newcommand{\T}{^{\mathsf{T}}}

% text macros
\newcommand{\todo}[1]{\textcolor{red}{#1}}  % gotta have \usepackage{xcolor} in main doc or this won't work
\newcommand{\acronym}[1]{{\small{#1}}}
\newcommand{\project}[1]{\textsl{#1}}
\newcommand{\RV}{\acronym{RV}}
\newcommand{\CRLB}{\acronym{CRLB}}

\setlength{\parindent}{1.4em} % trust in Hogg

\begin{document}\sloppy\sloppypar\raggedbottom\frenchspacing % trust in Hogg
\graphicspath{ {figures/} }
\DeclareGraphicsExtensions{.pdf,.eps,.png}

\title{Is spectro-perfectionism perfect?}

\begin{abstract}\noindent
Within the exoplanet community, spectrographs and their data reduction pipelines are designed with a primary goal in mind: obtaining the most precise radial velocity (\RV) measurement possible. 
In some cases, design decisions are made at the instrument level to sacrifice some degree of spectrograph throughput (?) for the sake of being able to extract a reliably calibrated spectrum from the data without losing much of the \RV\ information it encodes. 
The concept of ``spectro-perfectionism'' (cite) outlines a way to extract spectra without information loss from any reasonable spectrograph design, at least in principle. 
In this work we test the performances of traditional optimal extraction and of spectro-perfectionism on simulated spectra under a variety of spectrograph conditions. 
We find that under [x] conditions, spectro-perfectionism is able to extract precise \RV s [at/near the Cram\'er--Rao bound], outperforming optimal extraction [by a factor of...]. 
However, both methods begin to fail when [x], suggesting a future need for fully two-dimensional \RV\ extraction pipelines.
\end{abstract}

\section{Introduction}

Background on EPRV instruments and assumptions they make.

Background on optimal extraction and spectro-perfectionism

What we set out to do

\section{Data Generation}
how the data are generated, what the tunable knobs are, maybe a figure comparing fake data to e.g. MAROON-X test data or HARPS?

details on the time-series spectra we generate

\subsection{Theoretical Information Content}
since we're generating the data ourselves, it's possible to calculate the \CRLB\ on how much \RV\ information is encoded in the spectrograph images.

\section{Methods}
\subsection{Optimal Extraction}
implementation details

\subsection{Spectro-perfectionism}
implementation details, with some discussion of the trickiest parts (e.g. getting the A matrix right)

\section{Results \& Discussion}
Here's what we get when we run our two methods on data generated under a variety of conditions.

First we show that under a variety of spectrograph configurations, even ones where optimal extraction fails, spectro-perfectionism performs well.

Then we address the issues with implementing spectro-perfectionism in the real world: insert reasonable levels of noise into the algorithm and see where/how it breaks.

\section{Conclusions}

summarize results: conditions under which spectro-perfectionism is a good option, conditions under which optimal extraction will do, and conditions under which they both break

emphasize that even if both algorithms break, the \RV\ content is still in the 2D spectrograph images! future outlook on 2D modeling to get \RV s WITHOUT extracting a spectrum.



\bibliographystyle{apj}
\bibliography{}%general,myref,inprep}

\end{document}